%%%%%%%%%%%%%%%%%
% This is an sample CV template created using altacv.cls
% (v1.2, 11 February 2020) written by LianTze Lim (liantze@gmail.com). Now compiles with pdfLaTeX, XeLaTeX and LuaLaTeX.
%
%% It may be distributed and/or modified under the
%% conditions of the LaTeX Project Public License, either version 1.3
%% of this license or (at your option) any later version.
%% The latest version of this license is in
%%    http://www.latex-project.org/lppl.txt
%% and version 1.3 or later is part of all distributions of LaTeX
%% version 2003/12/01 or later.
%%%%%%%%%%%%%%%%

%% If you need to pass whatever options to xcolor
\PassOptionsToPackage{dvipsnames}{xcolor}

%% If you are using \orcid or academicons
%% icons, make sure you have the academicons
%% option here, and compile with XeLaTeX
%% or LuaLaTeX.
% \documentclass[10pt,a4paper,academicons]{altacv}

%% Use the "normalphoto" option if you want a normal photo instead of cropped to a circle
% \documentclass[10pt,a4paper,normalphoto]{altacv}

\documentclass[10pt,letterpaper,ragged2e]{altacv}

%% AltaCV uses the fontawesome and academicon fonts
%% and packages.
%% See http://texdoc.net/pkg/fontawesome and http://texdoc.net/pkg/academicons for full list of symbols. You MUST compile with XeLaTeX or LuaLaTeX if you want to use academicons.

% Change the page layout if you need to
\geometry{left=1.25cm,right=1.25cm,top=1.5cm,bottom=1.5cm,columnsep=1.2cm}

% The paracol package lets you typeset columns of text in parallel
\usepackage{paracol}

% Change the font if you want to, depending on whether
% you're using pdflatex or xelatex/lualatex
\ifxetexorluatex
  % If using xelatex or lualatex:
  \setmainfont{Lato}
\else
  % If using pdflatex:
  \usepackage[utf8]{inputenc}
  \usepackage[T1]{fontenc}
  \usepackage[default]{lato}
\fi

% Change the colours if you want to
\definecolor{Mulberry}{HTML}{2e40b4}
\definecolor{SlateGrey}{HTML}{2E2E2E}%2e40b4
\definecolor{LightGrey}{HTML}{666666}
\definecolor{Azul}{HTML}{3039c2}
\colorlet{heading}{Mulberry}
\colorlet{accent}{Azul}
\colorlet{emphasis}{SlateGrey}
\colorlet{body}{LightGrey}

% Change the bullets for itemize and rating marker
% for \cvskill if you want to
\renewcommand{\itemmarker}{{\small\textbullet}}
\renewcommand{\ratingmarker}{\faCircle}

%% sample.bib contains your publications
\addbibresource{sample.bib}
\usepackage{hyperref}
\begin{document}
\name{Fabián Francisco Ramírez Díaz}
\tagline{Estudiante de Ingeniería Civil Matemática\\
Especialidad Estadística}
%% You can add multiple photos on the left or right
\photoR{2.8cm}{s}
% \photoL{2.5cm}{Yacht_High,Suitcase_High}
\personalinfo{% 0
  % Not all of these are required!
  % You can add your own with \printinfo{symbol}{detail}
  \normalsize{
  \email{fabian.ramirez.di@gmail.com}
  \phone{+569-65497823}
  \mailaddress{Avenida Curauma Sur 1088}
  \location{Valparaíso, Chile}
  \homepage{\href{https://fabimath.github.io/Fabimath/}{https://fabimath.github.io/Fabimath/}}
  \github{\href{https://github.com/Fabimath}{https://github.com/Fabimath}}}
  %% You MUST add the academicons option to \documentclass, then compile with LuaLaTeX or XeLaTeX, if you want to use \orcid or other academicons commands.
  % \orcid{orcid.org/0000-0000-0000-0000}
}

\makecvheader

%% Depending on your tastes, you may want to make fonts of itemize environments slightly smaller
% \AtBeginEnvironment{itemize}{\small}

%% Set the left/right column width ratio to 6:4.
\columnratio{0.6}

% Start a 2-column paracol. Both the left and right columns will automatically
% break across pages if things get too long.
\begin{paracol}{2}

\cvsection{Experiencia}

\cvevent{Profesor asistente}{Universidad Técnica Federico Santa María}{2017 -- 2020}{Valparaíso}
\begin{itemize}
\item Ayudante de Cálculo Multivariado (2018,2019)
\item Ayudante de Ecuaciones Diferenciales Ordinarias (2018)
\item Ayudante de Teoría de Probabilidades y Procesos Estocásticos (2019)
\item Ayudante de Inferencia Estadística (2019)
\end{itemize}
\divider
\cvevent{Profesor asistente}{Universidad Adolfo Ibañez}{2020}{Viña del Mar}
\begin{itemize}
\item Profesor de laboratorio de probabilidad y estadística en R. (2020)
\item Profesor de laboratorio de estadística aplicada. (2020)
\item Ayudante de probabilidad y estadística. (2020)
\end{itemize}
\divider
\cvevent{Profesor Titular}{Preuniversitario Solidario Santa María}{2017 -- 2020}{Valparaíso}
\begin{itemize}
\item Profesor de cátedra de estudiantes de 4$^\circ$ Medio 
\item Profesor de cátedra de estudiantes de 3$^\circ$ Medio
\item Profesor de cátedra de plan intensivo
\end{itemize}



\cvsection{Educación}



\cvevent{Ingeniería Civil Matemática}{Universidad Técnica Federico Santa María}{Marzo 2016 -- Febrero 2020}{}
\begin{itemize}
\item Estudiante de ingeniería civil matemática mención estadística de quinto año.
\end{itemize}
\divider

\cvevent{Programa de intercambio CRUV}{Instituto de Estadística \\ Pontificia Universidad Católica de Valparaíso}{Septiembre 2019 -- Febrero 2020}{}
\begin{itemize}
\item Cursos de Data Science y series temporales.
\end{itemize}
\divider
\switchcolumn

\cvsection{Educación Online}
\cvevent{Moodle y Zoom}{}{}{}
\begin{itemize}
\item Experiencia y conocimiento sobre la plataforma Moodle para el proceso de educación y evaluación online.
\item Disposición, equipamiento y experiencia realizando clases vía Zoom.
\end{itemize}


\cvsection{Investigaciones}

\cvevent{COVID-19 en Chile}{DMAT - AM2V - CMM - CEPS}{Marzo 2020}{}
\begin{itemize}
\item Estudio y estimación de la tasa de asintomáticos.
\item Estimación de parámetros de modelos de ecuaciones diferenciales mediante Splines Penalizados.
\item Toda los resultados se envían directamente a los asesores de gobierno. \textcolor{blue}{\href{https://noticias.usm.cl/2020/04/13/matematicos-de-la-usm-colaboran-en-combate-contra-covid-19/?fbclid=IwAR3Q2RN1FoTV7S4Qd_dQiNR8Z27vTbPxvnm54iws63zprqQYu2CLeb3VCk4}{aquí}} para ver referencias.
\end{itemize}

\cvsection{Habilidades}

\cvtag{Análisis de Regresión}
\cvtag{Data Science}\\
\cvtag{Ecuaciones Diferenciales}
\cvtag{Álgebra Lineal}\\
\cvtag{Probabilidad y Estadística}\\

\divider\smallskip

\cvtag{Series de tiempo}
\cvtag{Geo-Estadística}\\
\cvtag{Estadística Espacio Temporal}\\
\cvtag{Aplicaciones de estadística en R}


\cvsection{Programación}

\cvskill{R}{5}
\divider

\cvskill{TeX}{5}
\divider

\cvskill{Python}{4}
\divider
%% Yeah I didn't spend too much time making all the
%% spacing consistent... sorry. Use \smallskip, \medskip,
%% \bigskip, \vpsace etc to make ajustments.
\medskip


\end{paracol}

\end{document}
